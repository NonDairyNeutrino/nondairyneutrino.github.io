\documentclass[11pt,letterpaper,twoside,titlepage]{book}

\usepackage[utf8]{inputenc}
\usepackage{amsmath}
\usepackage{amsfonts}
\usepackage{amssymb}
\usepackage{graphicx}
\usepackage[in]{fullpage}
\usepackage{parskip}
\usepackage{fancyhdr}

\pagestyle{fancy}
\fancyhf{}
\fancyhead[L]{\leftmark}
\fancyhead[R]{\thepage}

\usepackage{lmodern}
\usepackage[T1]{fontenc}
\newenvironment{dedication}
  {%\clearpage          % we want a new page
   \thispagestyle{empty}% no header and footer
   \vspace*{\stretch{1}}% some space at the top
   \itshape             % the text is in italics
   \raggedleft          % flush to the right margin
  }
  {\par % end the paragraph
   \vspace{\stretch{3}} % space at bottom is three times that at the top
   \clearpage           % finish off the page
  }
%dedication environment taken from 

\newcommand{\Mathematica}{\textit{Mathematica} }
\newcommand{\CB}{\emph{Constantly Busy} }

\author{Nathan Chapman}
\title{\Mathematica for the Mostly Busy}
\date{\today}

\begin{document}

	\maketitle
	
	\begin{dedication}
	
		I dedicate this work to the people who strive for truth about our reality, otherwise known as scientists.
		
		``Math: a lifetime of free entertainment''
		- Dr. Branko \'{C}urgus
	
	\end{dedication}
	
	\chapter*{Preface}
	
		This guide is meant to be the sequel to \emph{\Mathematica for the Constantly Busy}.  As the title might allude, this guide is for people who want to learn more about \Mathematica than just the basics.  That being said, here we will see some slightly more advanced topics both in \Mathematica and in math.  It will follow a similar format to the previous guide, so if you liked \CB, you hopefully will like this one.  If you did not like \CB, then you're out of luck because by the time you're reading this, I will have already wrote this and probably can't make changes for just you.
		
		Since this is the second part of a series, I will assume the reader has the equivalent knowledge of the information from \CB.  Therefore, I will not go over basic syntax like capitalizing functions, basic function format, and brackets.
		
\newpage
	
	\tableofcontents
		
	\part{Syntax}
	
%		\Mathematica does have some rather subtle and interestingly useful syntax that can be used in more advanced applications.  Namely things like rules, delayed rules, patterns, data type specification, etc.  When implemented correctly, even simple syntax like these can make a world of difference.
		
		\chapter{Definitions \& Assignments}
		
			Labeling a variable to be used later is one of the most key concepts in math and computer science, and that certainly isn't lost here.  The most basic variation of labeling an object, or \emph{setting} it as some other thing, can be used not only for numerical assignments, but also for graphical objects.  In other words, you can set a plot or a graphic to have a label of whatever you choose.  This is unbelievably helpful when dealing with results that contain A LOT of elements, especially when they are large sized plots, because when you want that set of plots, you can simply refer back to the label instead of either copying the actual output or needing to evaluate the expression again.  \Mathematica has a few different ways of labeling an object. 
		
			Here we will see more in-depth explanations of the different kinds of assignments in \Mathematica.
			
			\section{Set}
			
				To perform the most basic assignment operation we use the \textbf{Set} command, or simply \textbf{=}.  
				
				To set the variable $x$, or rather assign the label $x$ to the result that comes from the $expr$, we use
				
				\[ x = expr \]
				
				You can also set multiple labels $ x_1, x_2,\ldots $ to multiple objects $ expr_1, expr_2, \ldots $ at the same time with
				
				\[ \{x,y,\ldots\} = \{expr_1, expr_2, \ldots\} \]
				
				A very handy feature \Mathematica has, as well as other programs, is the ability to assign values and objects to certain elements in lists and/or arrays, etc.  
				
				To assign the label $a[[i]]$, that's the $i$-th element in the list/array with label $a$, to the object $v$ we use
				
				\[ a[[i]] = v \]
				
				\textbf{Note:} This is \textbf{NOT} the same as mathematical equality, which we will get to in the sections.  This is one of the biggest difficulties people encounter when they first start using \Mathematica.
				
			\section{Mathematical Equality}
			
				\Mathematica, much like other coding languages, has different syntax for mathematical equality.  In the case of denoting mathematical equality, we use the notation of a double-equals.  Rather, if the quantities $x,y$ are mathematically equal we denoted this relationship with 
				
				\[ x == y \]
				
				This notation can be used when providing conditions for the solve function, or simplifying/reducing an equation, etc.
			
			\section{Delayed Definition}
			
				An extra nifty feature \Mathematica has is the ability to delay an assignment until after a process has been evaluated.  This is the notion we use to define functions and objects that need to be reevaluated every time they are ran/called.
				
				For a variable $x$ to be defined by the result of process \emph{newly ran} $expr$, we use delayed definition by
				
				\[ x := expr \]
				
				\textbf{Note:}  There is a key difference between the delayed definition and the immediate ``set'' definition.  The key here is with the \emph{immediate} definition, the process $expr$ is ran, then the result is assigned to the object $x$ once, while with the \emph{delayed} definition the process $expr$ is called and evaluated when the object $x$ is called.
				
				For example, consider the code 
				
				\[ x = \mathrm{RandomReal}[] \]
				
				This will evaluate the RandomReal function and assign a psuedo-random real number between in the interval $[0,1]$ to the object $x$.  Then whenever $x$ is called, it will be replaced with this number.
				
				Now consider the code
				
				\[ x := \mathrm{RandomReal}[] \]
				
				This will assign the \emph{process} RandomReal to the object $x$, and therefore, every time the object $x$ is called, the the command RandomReal will run and output a different number each time (probably, since there is a nonzero probability it will output the same number twice).
				
				To further illustrate the point, the code
				
				\begin{center}
					x = RandomReal[]; \\
					y := RandomReal[]
				\end{center}
				
				followed by the the next cell with input
				
				\begin{center}
					Table[\{x,y\},\{4\}]//MatrixForm
				\end{center}
				
				will give the output
				
				\[ \left[
\begin{array}{cc}
 0.76451 & 0.296771 \\
 0.76451 & 0.992581 \\
 0.76451 & 0.958455 \\
 0.76451 & 0.658801 \\
\end{array}
\right]
\]

				This code calls the both $x,y$ four times and outputs the result in the form of a matrix.  It can be seen that each time $x$ is called, it is the same number, while each time $y$ is called, it is a different number.  Hopefully this is enough to get the point across.  If not, I highly suggest practicing and exploring the nature of this difference on your own before moving on to other content because this concept is used time and time again throughout application.
		
		\chapter{Functional Operations}
		
			\section{Functions}
		
				\Mathematica has a few different ways not only to use functions, but also to define them.  
				
				If we want to make a function $f(x) = x^2$, then we can do
				
				\[ f[x_]:= \#^2 \&[x] \]
				
				or 
				
				\[ f[x\_]:= \text{Function}[x,x^2] \]
				
				or
				
				\[ f[x\_]:= x \mapsto x^2 \]
				
				or 
				
				\[ f[x\_]:= x^2 \]
				
				\textbf{Note:} Here we use delay definition, :=, to make sure the function is properly defined, so when we call the function, it evaluates in the right order.  
				
				The first option is using a delayed definition to set a \textbf{pure function} as a function.  Usually we don't do this, but it is possible.  Normally, pure functions are used once in code instead of having to define a whole new named function first.  
				
				The second option is using the function function.  This is equivalent to the first option, except you can name your variables and the variable within the function function are treated as local.  
				
				The third option uses $\mapsto$ notation, which should be read ``mapsto''.  This is more of a ruling rather than a concrete functional programming way to do things.  
				
				The fourth option is how we normally define functions.  
				
				While all of these are equivalent, we usually only use pure functions and the last option for defining functions.
				
			\section{Map}
			
				Lets say you have a function $f$, and a set $\{ x_1, x_2, \dots, x_n \}$ and you want $\{ f(x_1), f(x_2), \dots, f(x_n)$.  \Mathematica has a built in function for just the task.  
				
				To thread a function, not necessarily mathematical, over a list simply use 
				
				\begin{center}
				
					Map[$f,list$]
				
				\end{center}
				
				or 
				
				\begin{center}
				
					$f /@ list$
				
				\end{center}
				
				This $/@$ command, as well as other shorthand syntactical features, is a nice, clean way of saying ``Map $f$ over $list$''.
				
				This is an incredibly useful function and makes it possible to create lists/sets of a function applied to each elements of a set, when the function isn't threadable from the start.  
				
				A great use of Map is to use it with pure functions.  For example
				
				\[ \#^2 \& /@\{ x_1, x_2, \dots, x_n \} = \{ x_1^2, x_2^2, \dots, x_n^2 \} \]
				
			\section{Apply}
			
				Apply kind of does the opposite of Map, in the sense Apply makes the entire list the argument of the function.  
				
				To apply the function $f$, again not necessarily mathematical, to the expression $expr$ we use
				
				\begin{center}
				
					Apply[$f, expr$]
				
				\end{center}
				
				or 
				
				\begin{center}
				
					$f$ @@ $expr$
				
				\end{center}
				
				What this really does is replace a thing that's called the ``head'' of $expr$, with $f$.
				
%			\section{MapThread}
%			
%				If you have a set of sets $\{ \{a,b,c\}, \{d,e,f\} \}$ and you want to apply a function $f$ to index-corresponding elements of each subset, there's a function for that.  
%				
%				To apply a function $f$ to index-corresponding elements of each subset of a multiple layered set $\{ \{a,b,c\}, \{d,e,f\} \}$, we use
%				
%				\begin{center}
%				
%					MapThread[$f, \{ \{a,b,c\}, \{d,e,f\} \}$]
%				
%				\end{center}
%				
%%				For example, say you have set of data from a satellite $SDSSData$ that contains the set of data of redshifts $V_{n,1}$, of length $L$, corresponding to different stars $V_n$.  Then you have functions $ar$ to take the average of redshift of all the objects.  Then we would use
%%				
%%				\begin{center}
%%				
%%					MapThread[$ar, V$]
%%				
%%				\end{center}
%%				
%%				which would give a set of length $L$ whose elements are the averages of 
		
		\chapter{Patterns}
		
			\Mathematica uses patterns to recognize things that have similar attributes.
		
			\section{Blank}
			
				We mostly use this pattern type to refer to variables in functions that can take any kind of argument.  
				
				For example if you want to create a function $f(x)$ where $x$ can be anything, e.g. number, symbol, list, we use 
				
				\begin{center}
				
					f[x\_]:= $expr$
				
				\end{center}
				
				This is usually how we do things, since you usually know what you're putting into the function.
				
			\section{Blank Sequence}
			
				If you want your function to only take multiple things of any kind, we use
				
				\begin{center}
				
					f[x\_\_]:= $expr$
				
				\end{center}
				
				We would use this in the case where our function is defined with other functions that only take lists as arguments. e.g. Length, Plus, etc.
				
			\section{Blank Null Sequence}
			
				If you want to be able to put any data type, of anything and treat it as a single variable we use 
				
				\begin{center}
				
					f[x\_\_\_]:= $expr$
				
				\end{center}
				
				This will take what ever you stipulate as the function argument, whether it be lists of lists and number and symbols, or whatever, \Mathematica will input that into $expr$ and evaluate it.
				
			\section{Alternatives}
			
				A handy way to do multiple things in one command is to use alternatives.  
				
				If you want to say to \Mathematica ``Hey, \Mathematica could you do me a favor and please replace every instance of $x$ with $y$ in $expr$?'', we would use
				
				\[ expr /. x|y->z \]
				
				We will see this $/.$ command used and defined again later.
				
			\section{Pattern}
			
				All of the previous sections have been about specific pre-defined patterns.  Now we will see how to make our own.  
			
				To define a general pattern $x$, with the pattern $obj$ we use
				
				\begin{center}
				
					\emph{x:obj}
					
				\end{center}
				
			\section{Except}
			
				If we want our pattern to be anything, except a specific thing, \Mathematica has got you covered.  
				
				To define a pattern \emph{x} to be anything except \emph{obj}, we use
				
				\begin{center}
				
					\emph{x}:Except[\emph{obj}]
				
				\end{center}
				
				Similarly, if you want to define your pattern \emph{x} to be \emph{obj}, but not some element in \emph{obj}, $ s \in obj $, we use
				
				\begin{center}
				
					\emph{x}:Except[\emph{s,obj}]
				
				\end{center}
				
			\section{Restrictions}
			
				To restrict your pattern to certain patterns with specific head, \Mathematica can do that as well.  
				
				To restrict a pattern \emph{x} to anything with the head \emph{head}, we use
				
				\begin{center}
				
					\emph{x\_head}
				
				\end{center}
				
			\section{Condition}
			
				\Mathematica not only has the ``If'' function, but also the feature to streamline the ``If'' process.  
				
				With the ``Condition'' function, you can apply your favorite ``If'' statements to any pattern \emph{patt}, or definition \emph{lhs := rhs} with
				
				\begin{center}
				
					\emph{patt /; test}
				
				\end{center}
				
				or 
				
				\begin{center}
				
					\emph{lhs:= rhs /; test}
				
				\end{center}
				
				respectively.
				
			\section{PatternTest}
			
				If you want to create a pattern, that also tests for some condition being true, there's already a function for that.  
				
				Say you want to create a pattern \emph{x} that matches to objects with the pattern \emph{p}, and also tests for the condition \emph{cond} being true, we use
				
				\begin{center}
				
					\emph{x:p?cond}
				
				\end{center}
				
				This pattern object, combined with the ``Cases'' function, is very similar to the ``Select'' function.  The difference between these two functions is the criteria/-on to witch the object is being tested.  I will leave an in-depth exploration as an exercise to the reader.
				
			\section{Optional}
			
				When using patterns in a function definition, we can stipulate if there are optional variables.  
				
				To define a function \emph{f} with a necessary variable \emph{x} of any pattern, and an optional variable \emph{y}, which if omitted defaults to \emph{z}, we use
				
				\begin{center}
				
					f[\emph{x\_,y\_:z}]:=\emph{expr}
				
				\end{center}
				
				where \emph{expr} is the machinery behind the function.
				
			\section{Default}
			
				If you want the same thing to happen for a bunch of variables when they are omitted, instead of needing to dictate what each one should do, we can set a default.  
				
				To set the default action \emph{x} for omitted variables in a function \emph{f}, we use
				
				\begin{center}
				
					Default[\emph{f}]=\emph{x}
				
				\end{center}
				
				To tell \Mathematica the variables \emph{x,y} are optional and should use the default action when omitted, we use
				
				\begin{center}
				
					f[\emph{x\_.,y\_.}]:=\emph{expr}
				
				\end{center}
				
			\section{Cases}
			
				If you have a list $list$ but you only want the elements of $list$ that meet a certain pattern $pat$, you use the Cases command with
				
				\begin{center}
				
					Cases[$list$,$pat$]
				
				\end{center}
				
				to give a list of the elements of $list$ that meet the pattern $pat$ in the same order they appear in $list$.
				
				\textbf{Note:} Here the distinction of the criterion being a pattern is important.  There are other functions that do the same operation for the criterion not being a pattern.
		
			\section{DeleteCases}
			
				While the Cases function give a list of elements that meet a pattern, the DeleteCases function does the complement of the operation.
				
				If you have a list $list$ but you only want the elements of $list$ that \textbf{\emph{do not}} meet a certain pattern $pat$, you use the DeleteCases function with
				
				\begin{center}
				
					DeleteCases[$list$,$pat$]
				
				\end{center}
				
				to give a list of the elements of $list$ that \textbf{\emph{do not}} meet the pattern $pat$ in the same order they appear in $list$.
			
			\section{Position}
			
				There will come a time when you have list and you want to know where in the list, the elements meeting a certain pattern are, or rather their position.  \Mathematica has you covered for that.
				
				If you have a list $list$ and want to know the position of the elements meeting a certain pattern $pat$, you use the Position function with
				
				\begin{center}
					
					Position[$list$, $pat$]
					
				\end{center}
				
%				\textbf{Corollary:}  To recreate the operation of Cases, simply input the code
%				
%				\begin{center}
%					
%						Table[$list$[[$k$]], \{k,Position[$list$,$pat$]\}]
%					
%				\end{center}

\pagebreak
			
			\section{Count}
			
				If you just want to know how many elements in your list $list$ meet a certain pattern $pat$, you use Count by 
				
				\begin{center}
					
						Count[$list$,$pat$]
					
				\end{center}
				
				This does the same process and therefore gives the same result as Length[Cases[$list$,$pat$]].
		
		\chapter{Rules}
		
	\part{Math}
	
%		In this part we will look at both more advanced mathematical topics, or rather go more in depth into such topics as in \CB, as well as how to implement them in \Mathematica.
		
		\chapter{Algebra}
		
			In this section we will encounter orthogonal functions, infinite sums and how to apply them in certain situations, and more.
			
			\section{Zeroes of a Function}
			
				In some instances you might find yourself with either with an extremely complicated equation, or even a transcendental (read ``analytically unsolvable'') equation that you need to end up solving numerically.  When this happens, \Mathematica has your back.  
				
				To find the root of a function $f$, or equation $f(x) = g(x)$, we use
				
				\begin{center}
				
					FindRoot[$f,\{x,x_0\}$]
				
				\end{center}
				
				or
				
				\begin{center}
				
					FindRoot[$lhs == rhs, \{x,x_0\}$]
				
				\end{center}
				
				\textbf{Note:} Since we're doing this numerically, we need to tell \Mathematica where to look.  This is where the $\{x,x_0\}$ comes from.  Basically we're asking \Mathematica ``Hey, \Mathematica, if you're not too busy could you please find the root of this function or equation for $x$ around $x_0$?''.
				
			\section{Solving Equations}
			
				There are two main methods in which to solve equations in \Mathematica, analytically and numerically.  In order to implement these methods, \Mathematica uses the functions \emph{Solve} and \emph{NSolve} respectively.
				
				\subsection{Solve}
				
					Given an \emph{algebraic} equation $ lhs = rhs $ where one, or both, of the sides is/are function(s) of a variable $x$
			
%			\section{Sums of Orthogonal Functions}
%			
%				Two functions, $f,g$, defined in an inner product space, $V$, are said to be orthogonal if the inner product, $<\cdot,\cdot>$, between them is zero, $<\cdot,\cdot> = 0$.  We will see how to identify such functions in the calculus chapter.  
%				
%				If we have a collection of orthogonal functions $\{f_n\}$, we can take their linear combination,$\sum f_n$ to form some really useful things.  This is analogous to vectors in $\mathbb{R}^n$, which reduces to the usual dot product or, for vectors $f = (f_1, f_2, \ldots, f_n ), g = (g_1, g_2, \ldots, g_n )$ in $\mathbb{R}^n$
%				
%				\[ <\vec{f},\vec{g}> = \sum_{i=1}^n f_i g_i \]
%				
%				Probably the most useful and famous of these linear combinations of orthogonal functions is over cosine and sine to form a Fourier series.
%				
%			\section{Fourier Series}
%			
%				The general Fourier series is defined as
%				
%				\[ f(x) \sim a_0 + \sum_{n=1}^\infty a_n \cos \left(\frac{n \pi}{L} x \right) + \sum_{n=1}^\infty b_n \sin \left( \frac{n \pi}{L} x \right) \]
%				
%				where $\sim$ denotes ``has the Fourier series'', and the coefficients $a_0, a_n, b_n$ are defined using calculus. Right now we're just concerned with the algebra behind it, namely the infinite series.  
%								
%				\textbf{Note:} There are some deeper mathematical subtleties behind both Fourier series and infinite series in general that we won't look at in this book. \\
%				
%				Here we are forming an infinite linear combination of the orthogonal functions $ \cos( \frac{n \pi}{L} x), \sin( \frac{n \pi}{L} x)$.  We will form a justification for the coefficients $a_0, a_n, b_n, \frac{n \pi}{L}$ in the calculus chapter.
%				
%			\section{Evaluating series}
%			
%				Once the Fourier series is found, the series within can be found to evaluate to explicit function or even numbers.  (In reality, this is when mathematicians say the series converges)
		
		\chapter{Calculus}
		
			In \CB the calculus chapter was very short because all it was, was how to do the derivative and integral functions.  Here, as with the rest of the book, we will look at calculus a little more in depth.
			
			\section{Derivatives}
			
				There are actually several different kinds of derivatives you can take in \Mathematica.  For instance you can take the total derivative of a function this is itself composed of several functions not of the same variable.
				
				\subsection{Partial Derivatives}
				
					Taking a partial derivative of a function in \Mathematica is just about the easiest thing to do. In \CB, I just referenced this function as a general derivative, but now we can elaborate some more.  There are several kinds of partial derivatives you can take, e.g. single derivatives, multiple derivatives, successive derivatives, vector derivatives (Gradient), etc.  
					
					\subsubsection{Single Derivatives}
					
						To take a single, partial derivative of the function $f$ with respect to $x$ in \Mathematica we use 
					
						\begin{center}
					
							D[$f,x$]
					
						\end{center}
					
						We can tweak this with the very useful, built-in commands to suit our needs.
					
					\subsubsection{Multiple Derivatives}
				
						To take a multiple, partial derivative of the function $f$ with respect to $x$ $n$ times in \Mathematica we use 
					
						\begin{center}
					
							D[$f,\{x,n\}$]
					
						\end{center}
					
						This is great if you only want the rate of change, curvature/rate of rate of change in one direction.  But, normally, there is more than one direction.
					
					\subsubsection{Successive Derivatives}
				
						To take a successive, partial derivatives of the function $f$ with respect to $x$, then $y$ in \Mathematica we use 
					
						\begin{center}
					
							D[$f,x,y$]
						
						\end{center}
					
						These three kinds are great if we're only looking at one value, but if we want to look at the behavior of the whole vector we can do that as well.
					
					\subsubsection{Vector Derivatives}
				
						To take a Euclidean-vector, derivative of the function $f$ with respect to $x_1, x_2, \dots, x_n$ in \Mathematica we use 
					
						\begin{center}
					
							D[$f, \{ \{x_1, x_2, \dots, x_n\} \}$]
					
						\end{center}
					
						\textbf{Note:} Here there are two sets of curly brackets enclosing the variables to which we are differentiating with respect to.  These are especially important, otherwise \Mathematica would think we are trying to differentiate $f$ with respect to $x_1, x_2$ times, and also there is a bunch of other stuff after that, that doesn't make sense so it would break anyways.  
						
						Later, in the vector analysis and  partial differential equations sections and chapters, we will see the sum of the second partial derivatives of a multivariate function, otherwise known as the Laplacian 
						
						\[ \nabla^2 f = \frac{\partial^2 f}{\partial x^2} + \frac{\partial^2 f}{\partial y^2} \]
					
				\subsection{Total Derivatives}
			
					\subsubsection{Single}
			
						The total derivative of a function $y = f(t,x_1(t),x_2(t),\dots,x_n(t))$ with respect to $t$ is defined as 
				
						\[ \frac{\partial y}{\partial t} = \frac{\partial f}{\partial t} + \sum_{k=1}^n \left[ f^\prime(x_{k,1..T} ) \prod_{i=1}^{T} x^\prime_{k,i..T}(x) \right] \]
				
						Where, with $(f \circ g)(x) = f(g(x))$,
				
						\[ f_{a..b} = f_a \circ f_{a+1} \circ \dots \circ f_{b-1} \circ f_b \]
				
						\textbf{Note:} This is when each argument is itself composed $T$ times with functions of $t$.  When each argument is only dependent on $t$ once, $T=1$, the above equation reduces to
				
						\[ \frac{\partial y}{\partial t} = \frac{\partial f}{\partial t} + \sum_{k=1}^n f'(x_k) x'_k(t) \]
				
						To take a total derivative of the function $f$ with respect to $t$ in \Mathematica we use 
					
						\begin{center}
					
							Dt[$f, t$]
					
						\end{center}
						
					\subsubsection{Multiple}
				
						To take a multiple, total derivative of the function $f$ with respect to $t, n$ times in \Mathematica we use 
					
						\begin{center}
					
							Dt[$f, \{t, n \}$]
					
						\end{center}
						
				\subsection{Options}
				
					There are a few options for D that can come in handy.
					
					\subsubsection{Differentiate with respect to a function}
					
						While not mathematically making too much sense, in \Mathematica you are able to differentiate a function with respect to another function, or rather a general variable.  
						
						We can differentiate the function $f(x) + f^\prime(x)$ with respect to the general variable 
						
						\begin{center}
						
							In: D[$f[x] + f^\prime[x], f[x]$]
							
							Out: 1
						
						\end{center}
						
					\subsubsection{Non Constant Variables}
					
						Sometimes you want to assume something about the variable in the function and what not.  
						
						If we want to differentiate the function $f(x,y)$, but assume $y$ might be a function of $x$, we do
						
						\begin{center}
						
							D[$f(x,y),x$, NonConstants $\rightarrow$ y]
						
						\end{center}
						
			\section{Integrals}
		
				In this version we will skip single variable indefinite and definite integrals, but we will see both integrals over a rectangle and over a region.  As with single variable integrals, one could use NIntegrate to numerically calculate integrals for which \Mathematica doesn't have an explicit representation.
				
				\subsection{Rectangular Region}
				
					Just like with other similar things, a definite integral over a rectangular region in $\mathbb{R}^n$ can be done very easily.  
					
					To integrate a function $f(x,y)$,
					
					\begin{center}

						Integrate[$f, \{x, x_{min}, x_{max} \}, \{y, y_{min}, y_{max} \}$]
					
					\end{center}
					
				\subsection{Any Region}
				
					To integrate the function $f(x,y)$ over any region, we use 
					
					\begin{center}

						Integrate[$f, \{x,y,\ldots\} \in reg$]
					
					\end{center}
					
					For example
					
					\begin{center}
					
						Integrate[Abs[Sqrt[x]],$ \{x,y\} \in $ Disk[]]
					
					\end{center}
					
					integrates the function $|\sqrt{x}|$ over the unit disk $\{(x,y) : x^2 + y^2 < 1\}$.
					
				\subsection{Options}
				
					For integrals, there are quite a few options.  
					
					\subsubsection{Assumptions}
					
						\Mathematica does its best to take into account of all the different possibilities for variables, like not restrict variable to be in the reals, but sometimes you know some of these restrictions are indeed true.  
						
						To integrate the function $f(x)$ with respect to $x$ but you know $x \in \mathbb{Z}$ (the integers), we can use
						
						\begin{center}
						
							Integrate[$f[x],x,$ Assumptions $\rightarrow x \in $ Integers]
						
						\end{center}
						
			\section{The Fourier Series}
			
				In the algebra chapter we saw the Fourier cosine and sine series
				
				\[ f(x) \sim a_0 + \sum_{n=1}^\infty a_n \cos \left(\frac{n \pi}{L} x \right) + \sum_{n=1}^\infty b_n \sin \left( \frac{n \pi}{L} x \right) \]
				
				Now we that we have derivatives and integrals, we can use this representation to do calculus on an otherwise stubborn function.  
				
				\textbf{Fun Fact:}  In the usual case of a Fourier series being the solution to a partial differential equation, the cosine and sine functions are called ``eigenfunctions'' of the corresponding boundary eigenvalue problem.
				
				\subsection{The Coefficients}
				
					The coefficients of the Fourier series above are defined as 
					
					\[ a_0 = \frac{1}{2L} \int_{-L}^L f(x) dx \]
					
					\[ a_n = \frac{1}{L} \int_{-L}^L f(x) \cos \left( \frac{n \pi}{L} x \right) dx \]
					
					\[ b_n = \frac{1}{L} \int_{-L}^L f(x) \sin \left( \frac{n \pi}{L} x \right) dx \]
					
					where $n \in \mathbb{N} $.  
					
					\Mathematica can usually calculate and use these integrals quite easily, even when you don't assume $n$ is in the integers.  To make it even quicker and easier for \Mathematica, you can tack on to the end ``Assumptions $\rightarrow n \in $ Integers''.
					
				\subsection{Using the Fourier Series}
				
					Now that we have the Fourier series, we can do some calculus, as well as other things, with it.  It should be noted there are special conditions that must be met to differentiate a Fourier series, but if those are met then \Mathematica can come in handy.  For integrals, there are no such conditions, so we can integrate a Fourier series without worry.  
					
					\textbf{Fun Fact:} We can differentiate and integrate Fourier series because the derivative and integral are each what's called a linear operator.  Because of this each operator is able to thread over each term in the series to turn the derivative of a sum into a sum of derivatives.  
					
					So, for example, say we have a piecewise smooth function $f(x)$ where $ -\pi < x \leq \pi $.  Then the Fourier coefficients are 
					
					\[ a_0 = \frac{1}{2\pi} \int_{-L}^L f(x) dx \]
					
					\[ a_n = \frac{1}{\pi} \int_{-L}^L f(x) \cos \left( n x \right) dx \]
					
					\[ b_n = \frac{1}{L} \int_{-L}^L f(x) \sin \left( n x \right) dx \]
					
					where $n \in \mathbb{N} $. Which gives the Fourier series
					
					\[ f(x) \sim a_0 + \sum_{n=1}^\infty a_n \cos (n x) + \sum_{n=1}^\infty b_n \sin (n x) \]
					
					Then, under the appropriate conditions, we can differentiate the series to yield
					
					\[ f^\prime(x) \sim -\sum_{n=1}^\infty a_n n \sin (n x) + \sum_{n=1}^\infty b_n n \cos (n x) \]
					
					and integrate to produce
					
%					\[ \int_{-\pi}&\pi f(x) dx \sim a_0 x + \sum_{n=1}^\infty a_n \frac{1}{n} \left[ \sin (n x) \right] + \sum_{n=1}^\infty b_n n \cos (n x) \]

			\section{The Fourier Transform}
			
				The Fourier series is defined for $n \in \mathbb{Z}$, which tells us our collection of eigenfunctions $\{f_n\}$ is indexed in the naturals; but what if we had an uncountably infinite collection of eigenfunctions?  Then we would need some sort of summation whose index is in an uncountably infinite set.  Sound familiar?  In the usual/physical case, this uncountably infinte indexing set is $\mathbb{R}$, the real numbers.  When the function is ``nice'', we get to use the good ol' fashion Riemann integral.  This turns our Fourier series of the function $f(x)$ into the Fourier Transform of $f(x)$
				
				\[ (\mathcal{F}f)(\xi) = \int_{-\infty}^\infty f(x) e^{-2\pi \imath x \xi } dx \]
				
			\section{The Laplace Transform}
			
				The Laplace transform is very closely related to the Fourier transform.  The Laplace transform is used heavily in solving differential equations, and leads to a great many interesting topics, such as the Gamma function $\Gamma(n)$.  
				
				The Laplace transform of a function $f(t), t \geq 0$ is defined as 
				
				\[ \mathcal{L}[ f(t) ](s) = \int_0^\infty f(t) e^{-s t} dt \]
					
			\section{The Dirac Delta Function}
			
				The Dirac delta function is ever so convenient and also ubiquitous in quantum mechanics and differential equations in general.  The Dirac delta ``function'',$\delta(x)$ is defined such that
				
				\[ \int_{-\infty}^\infty \delta(x) dx = 1 \]
				
				It can be thought of  as 
				
				\[ \delta(x) = \begin{cases}
								\infty & x = 0  \\
								0 & x \neq 0 \\
				 			\end{cases}
				\]
				
				though it is not rigorously defined this way.  
				
				In \Mathematica we implement this with
				
				\begin{center}
				
					DiracDelta[$x$]
				
				\end{center}
				
				Un-augmented, this only evaluates to a nonzero value at $x=0$, but we can simply translate it with 
				
				\begin{center}
				
					DiracDelta[$x-a$]
				
				\end{center}
				
				then the ``definition'' becomes
				
				\[ \delta(x) =  \begin{cases}
									\infty & x = a  \\
									0 & x \neq a \\
				 				\end{cases}
				\]
				
				The Dirac delta function is also defined for more than one dimension, which can be implemented with
				
				\begin{center}
				
					DiracDelta[$x_1, x_2, \ldots$]
				
				\end{center}
				
			\section{Vector Analysis}
			
				When doing vector analysis operations in \Mathematica, you first need to import the vector analysis package with
				
				\begin{center}
				
					Needs[``VectorAnalysis`'']
				
				\end{center}
				
				\textbf{Note:} The apostrophe at the end does need to be included or else the package will not be imported, and therefore the vector analysis operations will not be available.  
				
				\textbf{Note:} Unless you're working in \Mathematica 10, or later, the package does need to be imported.
				
				\subsection{Divergence}
				
					The divergence of a vector field $ \mathbf{F}(x_1, x_2, \ldots, x_n) = \left\langle f_1(x_1, x_2, \ldots, x_n), f_2(x_1, x_2, \ldots, x_n), \ldots, f_n(x_1, x_2, \ldots, x_n) \right\rangle $ can be defined as the inner product between the gradient operator
					
					\[ \nabla = \left\langle \frac{\partial}{\partial x_1}, \frac{\partial}{\partial x_2}, \ldots, \frac{\partial}{\partial x_n} \right\rangle \]
					
					and the vector field $ \mathbf{F}(x_1, x_2, \ldots, x_n) $ to yield
					
					\[ \nabla \cdot \mathbf{F} = \frac{\partial}{\partial x_1} f_1 + \frac{\partial}{\partial x_2} f_2 + \dots + \frac{\partial}{\partial x_n} f_n = \sum_{k = 1}^n \frac{\partial}{\partial x_k} f_k \]
					
					Then in \Mathematica, to take the divergence of the vector field $ \mathbf{F} = f(x_1, x_2, \ldots, x_n)$ in Cartesian coordinates, you use
					
%					\begin{center}
%					
%						Div[$\mathbf{F}(x_1, x_2, \ldots, x_n)$, $\{ x_1, x_2, \ldots, x_n \}$]
%						
%					\end{center}
%						
%					or
					
					\begin{center}
					
						Div[$\{ f_1, f_2, \ldots, f_n \}$, $\{ x_1, x_2, \ldots, x_n \}$]
					
					\end{center}
					
					To take the divergence of the vector field $\mathbf{F}$ in the non-Cartesian coordinate system $chart$, you use
					
					\begin{center}
					
						Div[$\mathbf{F}$, $\{ x_1, x_2, \ldots, x_n \}$, $chart$]
					
					\end{center}
					
					Examples of non-Cartesian coordinate systems are cylindrical coordinates, spherical coordinates, etc.
				
				\subsection{Curl}
				
					The curl of the three dimensional vector field $ \mathbf{F}(x_1, x_2, x_3) = \left\langle f_1(x_1, x_2, x_3), f_2(x_1, x_2, x_3), f_3(x_1, x_2, x_3) \right\rangle $ can be defined as the cross product between the gradient operator
					
					\[ \nabla = \left\langle \frac{\partial}{\partial x_1}, \frac{\partial}{\partial x_2}, \ldots, \frac{\partial}{\partial x_n} \right\rangle \]
					
					and the vector field $ \mathbf{F}(x_1, x_2, x_3) $ to yield
					
					\[ \nabla \times \mathbf{F} = \left\langle \frac{\partial}{\partial x_2} f_3 - \frac{\partial}{\partial x_3} f_2, \frac{\partial}{\partial x_3} f_1 - \frac{\partial}{\partial x_1} f_3, \frac{\partial}{\partial x_1} f_2 - \frac{\partial}{\partial x_2} f_1 \right\rangle \]
					
					Then in \Mathematica, to take the curl of the vector field $ \mathbf{F}(x_1, x_2, x_3) $ in Cartesian coordinates, you use
					
%					\begin{center}
%					
%						Curl[$\mathbf{F}(x_1, x_2, x_3)$, $\{ x_1, x_2, \ldots, x_n \}$]
%						
%					\end{center}
%						
%					or
					
					\begin{center}
					
						Curl[$\{ f_1(x_1, x_2, x_3), f_2(x_1, x_2, x_3), f_3(x_1, x_2, x_3) \}$, $\{ x_1, x_2, \ldots, x_n \}$]
					
					\end{center}
					
					To take the curl of the vector field $\mathbf{F}$ in the non-Cartesian coordinate system $chart$, you use
					
					\begin{center}
					
						Curl[$\mathbf{F}(x_1, x_2, x_3)$, $\{ x_1, x_2, \ldots, x_n \}$, $chart$]
					
					\end{center}
					
					Examples of non-Cartesian coordinate systems are cylindrical coordinates, spherical coordinates, etc.
			
				\subsection{Laplacian}
				
					The Laplacian of the scalar function $ f(x_1, x_2, \ldots, x_n) $ can be defined as the inner product between the gradient operator 
					
					\[ \nabla = \left\langle \frac{\partial}{\partial x_1}, \frac{\partial}{\partial x_2}, \ldots, \frac{\partial}{\partial x_n} \right\rangle \]
					
					and the gradient of the scalar function  $ f(x_1, x_2, \ldots, x_n) $,
					
					\[ \nabla f = \left\langle \frac{\partial}{\partial x_1}f, \frac{\partial}{\partial x_2}f, \ldots, \frac{\partial}{\partial x_n}f \right\rangle \]
					
					to yield
					
					\[ \nabla \cdot \nabla f = \frac{\partial^2}{\partial x_1^2}f + \frac{\partial^2}{\partial x_2^2}f + \dots + \frac{\partial^2}{\partial x_n^2}f = \sum_{k = 1}^n \frac{\partial^2}{\partial x_k^2} f \]
					
					Then in \Mathematica to take the Laplacian of a scalar function $f(x_1, x_2, \ldots, x_n)$ in Cartesian coordinates, you use
					
					\begin{center}
					
						Laplacian[$f$, $\{x_1, x_2, \ldots, x_n \}$]
					
					\end{center}
					
					To take the Laplacian of the scalar function $f(x_1, x_2, \ldots, x_n)$ in the non-Cartesian coordinate system $chart$, you use
					
					\begin{center}
					
						Laplacian[$f(x_1, x_2, \ldots, x_n \}$, $chart$]
					
					\end{center}
					
					Examples of non-Cartesian coordinate systems are cylindrical coordinates, spherical coordinates, etc.
			
		\chapter{Linear Algebra}
		
			\section{Arithmetic}
			
			\section{Inner Product}
			
			\section{Cross Product}
			
			\section{Outer Product}
			
			\section{Normalize}
			
			\section{Orthogonalize}
		
%			\section{Tensors}
	
		\chapter{Differential Equations}
		
			\section{Ordinary Differential Equations}
			
			\section{Partial Differential Equations}
	
	\part{Data}
	
		
	
	\part{Important Functions}
	
		\chapter{Plotting}
		
			Here we will look at more of the customization options for Plot.
			
			\section{Plot}
			
				\subsection{Options}
			
			\section{Plot3D}
			
			\section{VectorPlot}
			
			\section{VectorPlot3D}
			
			\section{StreamPlot}
			
		\chapter{Manipulate}
		
	\part{Graphics}
	
%		Here we will see some of the advanced options for graphics in general.
		
	\chapter*{Conclusion}
					
\end{document}